\documentclass[10pt,twocolumn]{article} 
\usepackage{simpleConference}
\usepackage{times}
\usepackage{graphicx}
\usepackage{amssymb}
\usepackage{url,hyperref}

\begin{document}

\title{Advanced Real-time Reconstruction Methods}

\author{Conner Brooks\\
\\
CAP 4453\\
\today \\
\\
University of Central Florida\\
Orlando, FL, USA\\
\\
connerbrooks@gmail.com\\
}

\maketitle
\thispagestyle{empty}

\begin{abstract}
  The paper by Whelan et al. presents a new SLAM--
  Simultaneous Localization and Mapping--system capable of producing high quality
  reconstructions of an 3 dimensional environment with a low cost RGB-D sensor
  e.g. the Microsoft Kinect \ref{1}. This system is made possible by three
  key techniques.
  First, General-Purpose computing on Graphics Processing Units (GPGPU) and a 3D 
  cyclical buffer trick to extend the scanning volume to and unbounded spatial
  region. Second, this system also overcomes pose estimation limitations by combining
  dense geometric and photometric camera pose constraints. Third, the map of the 
  environment is updated according to place recognition and loop closure constraints.
  In this paper the first two techniques will be covered.
\end{abstract}

\section{Introduction}
3D reconstruction of environments is an important problem for 
robotics, virtual reality, and augmented reality. Understanding the geometry
allows a robot to avoid obticals and navigate, and with augmented reality
allows for interesting interactions with the environment. The advent of
commodity depth sensors has resulted in a large amount of reasearch 
concerning 3D understanding of environments. The work by Whelan et al. is the 
culmination of much of that work and represents one of the most advanced 
3D scanning systems. To understand the work presented in this paper, one must
first understand the work in the seminal paper by Newcombe et al. 

% TODO finish general intro

\subsection{Understanding Depth Sensors}

\section{Kinect Fusion Basics}

\section{Extensions to Kinect Fusion}
\subsection{Cyclical Buffer Trick}
\subsection{Camera Pose Estimation}



\bibliographystyle{abbrv}
\bibliography{refs}
\end{document}
